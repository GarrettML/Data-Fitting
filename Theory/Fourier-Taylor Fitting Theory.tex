\documentclass{article}

\usepackage[margin=1in, paperheight=11in, paperwidth=8.5in]{geometry}

\usepackage{amsmath}

\begin{document}
\begin{table}[htbp]
\begin{tabular}{cl}
$c_h\left[q, h\right]$&Artificial output function harmonic coefficient of dimensions $q$ and $h$\\
$c_p\left[q, p\right]$&Artificial output function polynomial coefficient of dimensions $q$ and $p$\\
$f\left(t\left[q\right]\right) \forall q \in \left[1, Q\right]$&Artificial output parameter\\
$H$&Maximum degree of harmonic fitting\\
$h$&Artificial output function harmonic coefficient degree dimension index\\
$i$&Index of measurement instance\\
$P$&Maximum degree of polynomial fitting\\
$p$&Artificial output function polynomial coefficient degree dimension index\\
$Q$&Input parameter dimension count\\
$q$&Input paramater dimensional index\\
$t\left[q\right]$&Artificial input parameter of dimension $q$\\
$x\left[q, i\right]$&Measured input parameter of dimension $q$ at instance $i$\\
$y\left[i\right]$&Measured output parameter at instance $i$\\
$\delta\left[q, h\right]$&Phase offset of harmonic fitting of dimensions $q$ and $n$\\
$\varepsilon\left[i\right]$&Phase offset of harmonic fitting of dimensions $q$ and $n$\\
$\omega\left[q\right]$&Fundamental frequency of harmonic fitting of dimension $q$
\end{tabular}
\end{table}

\par{This document describes the formulation of an algorithm to select parameters for a function that is fitted to a measured data set. The family of functions that are generated by this algorithm are a linear combination of polynomial and harmonic terms. This may be considered as a hybrid of Taylor-series and Fourier-series fitting.}

\par{Assume we are given a set of measurements. We arbitrarily select one dimension of these measurements as our output parameter set, $y\left[i\right]$. The remainder are the input parameter set, $x\left[q, i\right]$.}
\par{The goal of the algorithm is to determine the coefficients of the function that best match the measure dataset.}
\par{The function has the form described in equation \eqref{eqn:functionform}}.
\begin{equation}
\label{eqn:functionform}
f\left(t\left[q\right]\right) = c_p\left[1, 0\right] + \sum\limits_{q = 1}^{q = Q} \left( \left( \sum\limits_{n = 1}^{p = P}{c_p\left[q, n\right] \cdot \left(t\left[q\right]\right)^{n}} \right) + \left( \sum\limits_{h = 1}^{h = H}{c_h\left[q, h\right] \cdot \cos\left(h \cdot \omega\left[q\right] \cdot t\left[q\right] + \delta\left[q, h\right]\right)} \right) \right)
\end{equation}
\par{$c_p\left[1, 0\right]$ is the static offset of the fitting function.}
\par{The error between the fitting function and the measured data is defined as in equation \eqref{eqn:fittingerror}.}
\begin{equation}
\label{eqn:fittingerror}
\varepsilon\left[i\right] = f\left(x\left[q, i\right]\right) - y\left[i\right]
\end{equation}
\par{The algorithm minimizes the net square error.}
\begin{subequations}
\label{eqn:errorminimization}
\begin{equation}
0 = \frac{\partial}{\partial c_h\left[q, h\right]} \left( \varepsilon^2\left[i\right] \right) = \varepsilon\left[ i \right] \cdot \frac{\partial \varepsilon\left[ i \right]}{\partial c_h\left[q, h\right]}
\end{equation}
\begin{equation}
0 = \frac{\partial}{\partial \delta\left[q, h\right]} \left( \varepsilon^2\left[i\right] \right) = \varepsilon\left[ i \right] \cdot \frac{\partial \varepsilon\left[ i \right]}{\partial \delta\left[q, h\right]}
\end{equation}
\begin{equation}
0 = \frac{\partial}{\partial \omega\left[q\right]} \left( \varepsilon^2\left[i\right] \right) = \varepsilon\left[ i \right] \cdot \frac{\partial \varepsilon\left[ i \right]}{\partial \omega\left[q\right]}
\end{equation}
\begin{equation}
0 = \frac{\partial}{\partial c_p\left[q, p\right]} \left( \varepsilon^2\left[i\right] \right) = \varepsilon\left[ i \right] \cdot \frac{\partial \varepsilon\left[ i \right]}{\partial c_p\left[q, p\right]}
\end{equation}
\end{subequations}
\par{Each of the partial derivatives in equation \eqref{eqn:errorminimization} are expanded in equation \eqref{eqn:partialderivativeexpansions}}
\begin{subequations}
\label{eqn:partialderivativeexpansions}
\begin{equation}
\frac{\partial \varepsilon\left[ i \right]}{\partial c_h\left[q, h\right]} = \cos\left( h \cdot \omega\left[q\right] x\left[q, i\right] + \delta\left[q, h\right] \right)
\end{equation}
\end{subequations}
\end{document}