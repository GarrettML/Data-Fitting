\documentclass{article}

\usepackage[margin=1in, paperheight=11in, paperwidth=8.5in]{geometry}

\usepackage{amsmath}

\usepackage{tabularx}

\usepackage[inline]{enumitem}

\title{Fourier-Taylor Fitting Algorithm Theory}
\author{Garrett Lim, garrett.m.lim@gmail.com}
\date{Last Update: \today}

\begin{document}
\maketitle

\section{Introduction}
\par{This document describes the formulation of an algorithm to select parameters for a function that is fitted to a measured data set. The family of functions that are generated by this algorithm are a linear combination of polynomial and harmonic terms. The function generated by this algorithm may be considered as a hybrid of Taylor series and a Fourier series.}

\section{Parameters}
\begin{table}[htbp]
\begin{tabularx}{\linewidth}{cXl}
Parameter&Description&Units\\
\hline
$c_h\left[q, h\right]$&Artificial output function harmonic coefficient of dimensions $q$ and $h$&[output unit]\\
$c_p\left[q, p\right]$&Artificial output function polynomial coefficient of dimensions $q$ and $p$&[output unit] / [input unit $q$]$^p$\\
$f\left(t\left[q\right]\right) \forall q \in \left[1, Q\right]$&Artificial output function&[output unit]\\
$H$&Maximum degree of harmonic fitting&Unitless\\
$h$&Artificial output function harmonic coefficient degree dimension index&Unitless\\
$N$&Number of measurements&Unitless\\
$n$&Index of measurement instance&Unitless\\
$P$&Maximum degree of polynomial fitting&Unitless\\
$p$&Artificial output function polynomial coefficient degree dimension index&Unitless\\
$Q$&Input parameter dimension count&Unitless\\
$q$&Input paramater dimensional index&Unitless\\
$t\left[q\right]$&Artificial input parameter of dimension $q$&[input unit $q$]\\
$x\left[q, n\right]$&Measured input parameter of dimension $q$ at instance $n$&[input unit $q$]\\
$y\left[n\right]$&Measured output parameter at instance $n$&[output unit]\\
$\delta\left[q, h\right]$&Phase offset of harmonic fitting of dimensions $q$&[radians]\\
$\varepsilon\left[n\right]$&Error of fitted function at instance $n$&[output unit]\\
$\omega\left[q\right]$&Fundamental frequency of harmonic fitting of dimension $q$&[radians] / [output unit]
\end{tabularx}
\end{table}

\section{Theory}
\par{Assume we are given a set of synchronous measurements. We may arbitrarily select one dimension of these measurements as our output parameter set, $y\left[n\right]$. The remainder are the input parameter set, $x\left[q, n\right]$.}

\par{The goal of the algorithm is to determine the coefficients of the function that best match the measure dataset. The function has the form described in equation \eqref{eqn:functionform}}.

\begin{subequations}
\label{eqn:functionform}
\begin{equation}
f\left(t\left[q\right]\right) = f_h\left(t\left[q\right]\right) + f_p\left(t\left[q\right]\right)
\end{equation}
\begin{equation}
f_h\left(t\left[q\right]\right) = \sum\limits_{q = 1}^{q = Q} \sum\limits_{h = 1}^{h = H}{c_h\left[q, h\right] \cdot \cos\left(h \cdot \omega\left[q\right] \cdot t\left[q\right] + \delta\left[q, h\right]\right)}
\end{equation}
\begin{equation}
f_p\left(t\left[q\right]\right) = c_p\left[1, 0\right] + \sum\limits_{q = 1}^{q = Q} \sum\limits_{p = 1}^{p = P}{c_p\left[q, n\right] \cdot \left(t\left[q\right]\right)^{n}}
\end{equation}
\end{subequations}

\par{The error between the fitted function and the measured data is defined as in equation \eqref{eqn:fittederror}.}

\begin{equation}
\label{eqn:fittederror}
\varepsilon\left[n\right] = \left(\sum\limits_{q = 1}^{q = Q}f\left(x\left[q, n\right]\right)\right) - y\left[n\right]
\end{equation}
\par{The algorithm minimizes the net square error.}
\begin{subequations}
\label{eqn:errorminimization}
\begin{equation}
0 = \frac{\partial}{\partial c_h\left[q, h\right]} \left( \varepsilon^2\left[n\right] \right) = \varepsilon\left[ n \right] \cdot \frac{\partial \varepsilon\left[ n \right]}{\partial c_h\left[q, h\right]}
\end{equation}
\begin{equation}
0 = \frac{\partial}{\partial \delta\left[q, h\right]} \left( \varepsilon^2\left[n\right] \right) = \varepsilon\left[ n \right] \cdot \frac{\partial \varepsilon\left[ n \right]}{\partial \delta\left[q, h\right]}
\end{equation}
\begin{equation}
0 = \frac{\partial}{\partial \omega\left[q\right]} \left( \varepsilon^2\left[n\right] \right) = \varepsilon\left[ n \right] \cdot \frac{\partial \varepsilon\left[ n \right]}{\partial \omega\left[q\right]}
\end{equation}
\begin{equation}
0 = \frac{\partial}{\partial c_p\left[q, p\right]} \left( \varepsilon^2\left[n\right] \right) = \varepsilon\left[ n \right] \cdot \frac{\partial \varepsilon\left[ n \right]}{\partial c_p\left[q, p\right]}
\end{equation}
\end{subequations}

\par{Each of the partial derivatives in equation \eqref{eqn:errorminimization} are expanded in equation \eqref{eqn:partialderivativeexpansions}. Note that each of the partial derivates in equation \eqref{eqn:partialderivativeexpansions} is two-dimensional or three-dimensional.}

\begin{subequations}
\label{eqn:partialderivativeexpansions}
\begin{equation}
\frac{\partial \varepsilon\left[ n \right]}{\partial c_h\left[q, h\right]} = \cos\left( h \cdot \omega\left[q\right] x\left[q, n\right] + \delta\left[q, h\right] \right)
\end{equation}
\begin{equation}
\label{eqn:phasepartialderivativeexpansion}
\frac{\partial \varepsilon\left[ n \right]}{\partial \delta\left[q, h\right]} = - c_h\left[q, h\right] \cdot \sin\left(h \cdot \omega\left[q\right] \cdot x\left[q, n \right] + \delta\left[q, h\right]\right)
\end{equation}
\begin{equation}
\frac{\partial \varepsilon\left[ n \right]}{\partial \omega\left[q\right]} = - c_h\left[q, h\right] \cdot h \cdot x\left[q, i \right] \cdot \sin\left(h \cdot \omega\left[q\right] \cdot x\left[q, i \right] + \delta\left[q, h\right]\right)
\end{equation}
\begin{equation}
\frac{\partial \varepsilon\left[ n \right]}{\partial c_p\left[q, p\right]} = \left\{\begin{array}{ccc}p = 0&:&1\\p\ne0&:&p \cdot \left( x\left[q, n \right] \right)^{\left( p - 1 \right)}\end{array}\right.
\end{equation}
\end{subequations}

\par{The partial derivatives of the harmonic series are problematic because they incorporate coefficients. For this reason, it is necessary to \begin{enumerate*}[label=(\alph*)]\item{determine the fundamental frequencies, $\omega[q]$, in advance, and}\item{dissociate the phase and amplitude coefficients}\end{enumerate*}. The algorithm assumes that these are the first $q$ fundamental frequencies observed in $y$, as computed from a Fourier transformation.}

\par{The algorithm associates each dimension with its fundamental frequency by computing the minimum error of the simplified fitted function in equation \eqref{eqn:simplifiedfittedfunction}.}

\begin{equation}
\label{eqn:simplifiedfittedfunction}
f\left[n\right] = A \cdot \cos\left( \omega \cdot x\left[q, n\right] \right) + B \cdot \sin\left( \omega \cdot x\left[q, n\right] \right)
\end{equation}

\par{Multiple dimensions may share a fundamental frequency.}

\par{Note that computation of the dimension-associated fundamental frequencies is computationally expensive. If you, the reader, can find a way to improve this aspect of the algorithm, please contact the author.}

\par{The phase and amplitude coefficients can be separated via trigonometric identity.}

\begin{equation}
C \cos \left( \theta + \delta \right) = C \cos\theta \cos\delta - C \sin\theta \sin\delta = A \cos\theta + B \sin\theta
\end{equation}

\begin{equation}
C = \sqrt{A^2 + B^2}
\end{equation}

\begin{equation}
\tan\delta = - \frac{B}{A}
\end{equation}

\par{The partial derivatives in equation \eqref{eqn:errorminimization} can be adjusted to those in equation \eqref{eqn:adjustederrorminimization}. Remember that the fundamental frequencies, $\omega\left[q\right]$, are determined by a subprocess of the main algorithm.}

\begin{subequations}
\label{eqn:adjustederrorminimization}
\begin{equation}
\frac{\partial \varepsilon\left[ n \right]}{\partial a_h\left[q, h\right]} = \cos\left( h \cdot \omega\left[q\right] x\left[q, n\right] \right)
\end{equation}
\begin{equation}
\frac{\partial \varepsilon\left[ n \right]}{\partial b_h\left[q, h\right]} = \sin\left( h \cdot \omega\left[q\right] x\left[q, n\right] \right)
\end{equation}
\begin{equation}
\frac{\partial \varepsilon\left[ n \right]}{\partial c_p\left[q, p\right]} = \left\{\begin{array}{ccc}p = 0&:&1\\p\ne0&:&p \cdot \left( x\left[q, i \right] \right)^{\left( p - 1 \right)}\end{array}\right.
\end{equation}
\end{subequations}

\par{The coefficients can be reorganized into a one-dimensional parameter, $\gamma$, that is constructed by the algorithm.}

\begin{equation}
\label{eqn:onedimensionalcoefficients}
\gamma\left[s\right] = \left\{\begin{array}{c}
	c_p\left[q = 1, p = 0\right]\\
	c_p\left[q = 1, p = 1\right]\\
	\vdots\\
	c_p\left[q, p = P[q]\right] : q = 1\\
	c_p\left[q, p = 1\right] : q = 2\\
	\vdots\\
	c_p\left[q, p = P\left[q\right] : q = 2\right]\\
	\vdots\\
	c_p\left[q, p = P\left[q\right]\right] : q = Q\\
	a_h\left[q = 1, h = 1\right]\\
	b_h\left[q = 1, h = 1\right]\\
	\vdots\\
	a_h\left[q, h = H\left[q\right]\right] : q = 1\\
	b_h\left[q, h = H\left[q\right]\right] : q = 1\\
	\vdots\\
	a_h\left[q, h = H\left[q\right]\right] : q = Q\\
	b_h\left[q, h = H\left[q\right]\right] : q = Q
\end{array}\right\}
\end{equation}

\par{Note that in equation \eqref{eqn:onedimensionalcoefficients} we allow that the maximum fitting degrees, $H[q]$ and $P[q]$, may vary by dimension, $q$. The number of coefficients must not exceed the number of measurement instances.\footnote{i.e., $1 + \sum\limits_{q = 1}^{q = Q}\left(1 + 2H\left[q\right] + P\left[q\right]\right) \le I$.}}

\par{The variables can also be reorganized into a conformal two-dimensional parameter, $\xi$, as in equation \eqref{eqn:twodimensionalvariables}.}

\begin{equation}
\label{eqn:twodimensionalvariables}
\xi\left[s, n\right] = \left\{\begin{array}{c}
	1\\
	x^1\left[q, n\right] : q = 1\\
	\vdots\\
	x^{P\left[q\right]}\left[q, n\right] : q = 1\\
	x^1\left[q, n\right] : q = 2\\
	\vdots\\
	x^{P\left[q\right]}\left[q, n\right] : q = 2\\
	\vdots\\
	x^{P\left[q\right]}\left[q, n\right] : q = Q\\
	\cos\theta\left[q, h, n\right] : h = 1, q = 1\\
	\sin\theta\left[q, h, n\right] : h = 1, q = 1\\
	\vdots\\
	\cos\theta\left[q, h, n\right] : h = H\left[q\right], q = 1\\
	\sin\theta\left[q, h, n\right] : h = H\left[q\right], q = 1\\
	\vdots\\
	\cos\theta\left[q, h, n\right] : h = H\left[q\right], q = Q\\
	\sin\theta\left[q, h, n\right] : h = H\left[q\right], q = Q
\end{array}\right\} : \theta\left[q, h, n\right] = h \cdot \omega\left[q\right] \cdot x\left[q, n\right]
\end{equation}

\par{We can use these linearized coefficients and variables to compute the fitted functional values by a different manner.}

\begin{equation}
\sum\limits_{q = 1}^{q = Q}f\left(x\left[q, n\right]\right) = \gamma\left[s\right] \cdot \xi\left[s, n\right]
\end{equation}

\par{The partial derivatives of the error can be conveniently redefined as well.}

\begin{subequations}
\label{eqn:partialderivativeexpansions}
\begin{equation}
\left( \frac{\partial \varepsilon\left[ n \right]}{\partial \gamma\left[s\right]} : \gamma\left[s\right] = a_h\left[q, h\right] \right) = \cos\left( h \cdot \omega\left[q\right] \cdot x\left[q, n\right] \right)
\end{equation}
\begin{equation}
\left( \frac{\partial \varepsilon\left[ n \right]}{\partial \gamma\left[s\right]} : \gamma\left[s\right] = b_h\left[q, h\right] \right) = \sin\left( h \cdot \omega\left[q\right] \cdot x\left[q, n\right] \right)
\end{equation}
\begin{equation}
\left( \frac{\partial \varepsilon\left[ n \right]}{\partial \gamma\left[s\right]} : \gamma\left[s\right] = c_p\left[q, p\right] \right) = \left\{\begin{array}{ccc}p = 0&:&1\\p\ne0&:&p \cdot x^{\left( p - 1 \right)}\left[q, n \right]\end{array}\right.
\end{equation}
\end{subequations}

\par{These can be used to compute the coefficients associated with minimum net error.}

\begin{equation}
y\left[n\right] \cdot \frac{\partial \varepsilon\left[ n \right]}{\partial \gamma\left[s_2\right]} = \left( \gamma\left[s_1\right] \cdot \xi\left[s_1, n\right] \right) \cdot \frac{\partial \varepsilon\left[ n \right]}{\partial \gamma\left[s_2\right]}
\end{equation}

\par{At this point, the algorithm filters the data to exclude instances with incomplete data. Such a filtration will exclude instances, $n$, with null output measurements, $y\left[n\right]$, and null input measurements, $\xi\left[s, n\right]$.}

\begin{equation}
\left[\xi\left[s_1, n\right] \cdot \frac{\partial \varepsilon\left[ n \right]}{\partial \gamma\left[s_2\right]}\right] \cdot \gamma\left[s_1\right] = \frac{\partial \varepsilon\left[ n \right]}{\partial \gamma\left[s_2\right]} \cdot y\left[n\right]
\end{equation}

\par{Note that $\left[\xi \cdot \frac{\partial \varepsilon}{\partial \gamma}\right]$ is a square matrix. The algorithm then uses Gaussian elimination to compute the coefficients, $\gamma$.}

\par{Finally, the amplitudes, $c_h$, and phase offsets, $\delta_h$, of the harmonic series are computed from the isolated amplitudes, $a_h$ and $b_h$.}

\begin{subequations}
\begin{equation}
c_h\left[q, h\right] = \sqrt{a_h^{2}\left[q, h\right] + b_h^{2}\left[q, h\right]}
\end{equation}
\begin{equation}
\tan\left(\delta\left[q, h\right]\right) = \frac{b_h\left[q, h\right]}{a_h\left[q, h\right]}
\end{equation}
\end{subequations}
\end{document}